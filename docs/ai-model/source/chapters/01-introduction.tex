% --------------------------------------------------------------------------------------------------
% Section: Introduction
% --------------------------------------------------------------------------------------------------

\section{Introduction}

% Contextual background on the significance of lung cancer and motivation for early detection
Lung cancer is one of the most lethal cancers globally, with a high mortality rate primarily caused 
by late detection. Early and accurate diagnosis plays a pivotal role in improving treatment outcomes 
and patient survival. Recent advances in artificial intelligence, particularly in deep learning, 
have paved the way for automated systems that can assist in medical image analysis with a 
performance nearing that of expert radiologists.

% General description of the project and what it aims to solve, introducing the use of CNNs
This project introduces a multi-stage deep learning pipeline designed to support the detection, 
classification, and localization of lung abnormalities in axial slices from chest CT scans. The 
pipeline employs 2D Convolutional Neural Networks (CNNs) in a modular architecture to replicate the 
diagnostic process used by clinicians.

% List of components in the proposed system, explained clearly and sequentially
The pipeline is composed of the following key components:

\begin{enumerate}
    % First model: does the initial categorization of CT slices into Normal, Benign, or Malignant
    \item \textbf{Primary Classification Model:} A multi-class CNN that analyzes a given CT slice 
    and classifies it into one of three categories: \emph{Normal}, \emph{Benign}, or 
    \emph{Malignant}. This model acts as the entry point of the pipeline and quickly filters out 
    normal cases.

    % Second model: only runs on positive cases (Benign/Malignant) to find the tumor's location
    \item \textbf{Tumor Localization Model:} If the classification result is \emph{Benign} or 
    \emph{Malignant}, the image is passed to a secondary CNN that performs object localization. This 
    model predicts bounding box coordinates that delineate the tumor area within the image, thus 
    providing spatial context and visual support for clinical interpretation.

\end{enumerate}

% Benefits of combining classification and localization for real-world usage and diagnostics
By combining classification and localization, the system not only flags potentially pathological 
cases but also highlights the specific region of interest, making it a valuable decision-support 
tool for radiologists. Each model is trained and validated independently to optimize its performance 
and ensure robustness across diverse imaging conditions.

% Closing statement for the Introduction section; previews what the document will cover next
This modular design allows for flexible updates, such as replacing or fine-tuning individual 
components as more data becomes available or clinical needs evolve. Future sections will describe 
the dataset, preprocessing techniques, augmentation strategies, model architectures, training 
methodologies, evaluation metrics, and experimental results.
