% --------------------------------------------------------------------------------------------------
% Evaluation Section
% --------------------------------------------------------------------------------------------------

\section{Evaluation}

% --------------------------------------------------------------------------------------------------
% Subsection: Metrics Used
% --------------------------------------------------------------------------------------------------

\subsection{Evaluation Metrics}

% Explain the importance of evaluation metrics in model performance assessment
To assess the performance of the lung cancer detection model based on ResNet50, several standard 
evaluation metrics were used. These metrics provide a comprehensive overview of how well the model 
performs on the task of distinguishing between multiple lung cancer classifications.

% List of commonly used metrics with definitions and formulas
\begin{itemize}
    % Accuracy: Overall correctness of the model
    \item \textbf{Accuracy} – the proportion of correct predictions (both true positives and true 
    negatives) among the total number of predictions. It is defined as:
    \[
    \text{Accuracy} = \frac{TP + TN}{TP + TN + FP + FN}
    \]

    % AUC: Discriminative capability of the model
    \item \textbf{AUC (Area Under the Curve)} – measures the ability of the classifier to 
    distinguish between classes. It represents the area under the Receiver Operating Characteristic 
    (ROC) curve. Higher AUC indicates better performance.

    % Sensitivity: Ability to detect positive cases correctly
    \item \textbf{Sensitivity (Recall or True Positive Rate)} – the proportion of actual positives 
    correctly identified by the model. It is calculated as:
    \[
    \text{Sensitivity} = \frac{TP}{TP + FN}
    \]

    % Specificity: Ability to detect negative cases correctly
    \item \textbf{Specificity (True Negative Rate)} – the proportion of actual negatives correctly 
    identified by the model. It is calculated as:
    \[
    \text{Specificity} = \frac{TN}{TN + FP}
    \]
\end{itemize}

% --------------------------------------------------------------------------------------------------
% Subsection: Confusion Matrix Table
% --------------------------------------------------------------------------------------------------

\subsection{Sample Results}

% Present the confusion matrix as a result of model testing
After training the ResNet50 model on the lung cancer dataset and evaluating on the test set, we 
obtained the following confusion matrix:

% Confusion Matrix Table
\begin{table}[H]
    \centering
    \begin{tabular}{|c|c|c|c|}
        \hline
        \textbf{} & \textbf{Predicted Benign} & \textbf{Predicted Malignant} & \textbf{Predicted Normal} \\
        \hline
        \textbf{Actual Benign} & 165 & 1 & 14 \\
        \hline
        \textbf{Actual Malignant} & 3 & 162 & 16 \\
        \hline
        \textbf{Actual Normal} & 5 & 0 & 177 \\
        \hline
    \end{tabular}
    \caption{Confusion matrix of model predictions across three classes.}
\end{table}

% --------------------------------------------------------------------------------------------------
% Subsection: Metric Values Derived from the Table
% --------------------------------------------------------------------------------------------------

\subsection{Derived Metric Values}

% Compute evaluation metrics using values from the confusion matrix
Based on the confusion matrix shown above, we compute the following metric values (reported by the system):

\begin{itemize}
    % Accuracy calculation
    \item \textbf{Accuracy:} 0.9282
    % Sensitivity calculation
    \item \textbf{Sensitivity:} 0.9281
    % Specificity calculation
    \item \textbf{Specificity:} 0.9640
    % AUC value reported from ROC analysis
    \item \textbf{AUC:} 0.9913
\end{itemize}

% --------------------------------------------------------------------------------------------------
% Subsection: Final Evaluation Summary
% --------------------------------------------------------------------------------------------------

\subsection{Conclusion}

% Summarize model performance with references to key metrics
The evaluation metrics demonstrate that the fine-tuned ResNet50 model performs strongly in 
multi-class lung cancer classification. High sensitivity and specificity indicate effective 
detection of both positive and negative cases across all classes. Additionally, the macro AUC score 
of 0.9913 reflects excellent discriminative ability, confirming the model's potential suitability 
for clinical diagnostic assistance.

% Link performance with relevant SDGs
Furthermore, this research contributes to the achievement of the United Nations Sustainable 
Development Goals (SDGs), specifically:

\begin{itemize}
    \item \textbf{Goal 3:} Ensure healthy lives and promote well-being for all at all ages — by 
    supporting early and accurate detection of lung cancer.
    \item \textbf{Goal 9:} Build resilient infrastructure, promote inclusive and sustainable 
    industrialization and foster innovation — by leveraging AI technologies for advanced healthcare 
    diagnostics.
    \item \textbf{Goal 17:} Strengthen the means of implementation and revitalize the global 
    partnership for sustainable development — by encouraging collaborative research and sharing open 
    medical AI tools across borders.
\end{itemize}
