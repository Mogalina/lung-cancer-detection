% --------------------------------------------------------------------------------------------------
% Section: Implementation
% --------------------------------------------------------------------------------------------------

\section{Implementation}

% Brief introduction to ResNet, focusing on its innovation: residual connections to solve vanishing 
% gradients.
\textbf{ResNet}, introduced by Kaiming He et al. in 2015, is a deep convolutional neural network 
architecture designed to overcome the vanishing gradient problem that hampers training of very deep 
networks. It achieves this by using residual learning through shortcut (skip) connections that 
allow gradients to flow directly across layers, enabling stable training of networks with dozens or 
even hundreds of layers without performance degradation. \cite{gb2023}

% Justification for using ResNet in a medical context.
This architecture has been widely adopted in medical imaging tasks, including lung cancer detection, 
due to its ability to learn complex hierarchical features from images while maintaining training 
stability and accuracy. \cite{jptcp2023}

% --------------------------------------------------------------------------------------------------
% Subsection: Environment and Tools
% --------------------------------------------------------------------------------------------------

\subsection{Environment and Tools}
% Details about development environment, choice of language/framework, and hardware limitations.
The implementation of the lung cancer detection system was carried out using the \textit{Python 
programming language}, chosen for its readability and wide support in the machine learning 
community. The model was developed and trained using the \textit{PyTorch framework}, known for its 
dynamic computation graph and intuitive design. All experiments were conducted on a macOS 
environment using CPU-based processing, as no dedicated GPU was available. Despite the computational 
limitations, careful model optimization and efficient batch processing allowed for manageable 
training times. The overall environment provided a stable and flexible foundation for implementing 
and testing the ResNet50-based multiclass classification model.

% --------------------------------------------------------------------------------------------------
% Subsection: Model Setup
% --------------------------------------------------------------------------------------------------

\subsection{Model Setup}
% Overview of how the ResNet50 model was loaded and adapted to the custom task.
The model architecture for this project was based on \textbf{ResNet50}, a 50-layer deep 
convolutional neural network known for its use of residual connections to enable training of very 
deep networks. Using PyTorch’s \textit{torchvision.model}, a pre-trained version of ResNet50 was 
loaded, leveraging weights learned from the given dataset.

\begin{itemize}
    % Explains the basic building blocks for feature extraction in CNNs.
    \item \textbf{Convolutional Layers:} In ResNet, convolutional layers are organized within 
    residual blocks and include a combination of 1×1 and 3×3 convolutions (bottleneck blocks) to 
    efficiently capture complex features while controlling computational cost. They form the 
    fundamental building blocks for feature extraction in the network.

    % Key innovation of ResNet: shortcut connections for residual learning.
    \item \textbf{Residual Blocks:} The hallmark of ResNet, residual blocks add the input of a block 
    directly to its output via shortcut connections. This residual learning framework allows the 
    network to learn the difference (residual) between the input and desired output, which helps 
    prevent the vanishing gradient problem and enables training of very deep networks with improved 
    accuracy and convergence.
    
    % Describes the use and benefit of max pooling to reduce spatial dimensions.
    \item \textbf{Pooling Layer:} After the first 7×7 convolution layer and ReLU activation, a 3×3 
    max pooling with stride 2 is applied. Pooling reduces the number of parameters, adds translation 
    invariance, and helps control overfitting.

    % BatchNorm standardizes outputs, improving training speed and stability.
    \item \textbf{Batch Normalization Layer:} Batch normalization is applied after each 
    convolutional layer and before the activation (ReLU). It helps to standardize the inputs to each 
    layer (zero mean and unit variance), making the model more stable.

    % ReLU provides non-linearity and is applied after each BatchNorm layer.
    \item \textbf{Activation Functions:} The primary activation function used is ReLU (Rectified 
    Linear Unit), defined as \textit{f(x) = max(0, x)}. ReLU is applied after each Batch 
    Normalization layer in each convolutional block.

    % The final classification layer adapted to a 3-class problem instead of ImageNet's 1000.
    \item \textbf{Fully Connected (Dense) Layer:} After all the convolutional and pooling layers, 
    the output feature map is flattened into a vector. This vector is passed to a fully connected 
    layer. The original ResNet50 has a Dense layer with 1000 outputs (for the 1000 ImageNet classes) 
    followed by a softmax activation, but in our case we replaced the final dense layer with one 
    that outputs 3 neurons, each representing a class (benign, malignant, normal).
\end{itemize}
