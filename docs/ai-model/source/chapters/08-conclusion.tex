% --------------------------------------------------------------------------------------------------
% Section: Conclusion of the Paper
% --------------------------------------------------------------------------------------------------

\section{Conclusion}

% --------------------------------------------------------------------------------------------------
% Subsection: Summary of outcomes of the project
% --------------------------------------------------------------------------------------------------

\subsection{Summary of Outcomes}

% High-level context and goal of the project
Lung cancer remains one of the leading causes of cancer-related mortality worldwide, largely due to 
the difficulty of early detection. This project addresses this critical challenge by introducing a 
modular deep learning pipeline that supports the automated detection, classification, and 
localization of lung abnormalities using axial slices from chest CT scans.

% Brief overview of the system architecture (two stages)
The system is built around 2D Convolutional Neural Networks (CNNs) and is composed of two major stages:

\begin{enumerate}
    % First stage: classification
    \item \textbf{Primary Classification Model:} A multi-class CNN that categorizes CT slices into 
    one of three classes: \textit{Normal}, \textit{Benign}, or \textit{Malignant}. This model 
    filters out normal cases and prioritizes potentially pathological slices for further analysis.
    
    % Second stage: localization for abnormal cases
    \item \textbf{Tumor Localization Model:} For cases identified as \textit{Benign} or 
    \textit{Malignant}, a second CNN performs spatial localization by predicting bounding box 
    coordinates around the suspected tumor region. This output supports radiologists by highlighting 
    the exact area of concern.
\end{enumerate}

% Description of the modular architecture and dataset used
Each component is independently trained and validated, allowing the system to remain flexible and 
robust. The modular nature of the pipeline ensures that individual models can be updated, retrained, 
or replaced without affecting the entire system.

% Dataset details
The pipeline was trained and evaluated using the publicly available \textbf{IQ-OTH/NCCD Lung Cancer 
Dataset}, obtained from The Cancer Imaging Archive (TCIA). The dataset includes diverse CT scans 
from patients of varying demographics and cancer stages, captured across multiple slices in axial 
view. This diversity helped improve model generalization and applicability across real-world 
scenarios.

% Wrap-up statement about the system’s potential
Overall, the project demonstrates that CNN-based systems can effectively assist in the early 
detection of lung cancer, offering both classification and localization capabilities that are 
valuable in a clinical decision-support context.

% --------------------------------------------------------------------------------------------------
% Subsection: Future directions and possible enhancements
% --------------------------------------------------------------------------------------------------

\subsection{Future Improvements}

% Description of planned or proposed technical upgrades
While the current implementation demonstrates encouraging results, several enhancements can further 
improve the model’s utility and performance:

\begin{itemize}
    % Performance improvement by GPU acceleration
    \item \textbf{GPU-Accelerated Training:} The current implementation is limited to CPU-based 
    training, which significantly increases computation time. Future iterations should leverage GPU 
    acceleration to drastically reduce training time and enable experimentation with larger models 
    or 3D volumetric inputs.
    
    % Adding temporal/spatial context using 3D volume input
    \item \textbf{3D CT Volume Integration:} The pipeline currently processes individual 2D slices. 
    Integrating 3D CNNs could improve spatial coherence and provide a more holistic understanding 
    of tumor structure across adjacent slices.
    
    % Clinical relevance and real-world testing
    \item \textbf{Clinical Validation:} Collaborating with radiologists and deploying the pipeline 
    in a clinical simulation setting could provide feedback for refining the system and identifying 
    failure cases.
    
    % Real-world deployment integration ideas
    \item \textbf{Pipeline Automation:} Future work could explore integrating automated DICOM 
    preprocessing, PACS compatibility, and real-time inference modules to enable a fully integrated 
    clinical tool.
\end{itemize}

% --------------------------------------------------------------------------------------------------
% Subsection: Sustainability relevance of the project
% --------------------------------------------------------------------------------------------------

\subsection{Sustainable Development Goals (SGDs)}

% High-level justification aligning the project with SDGs
The implementation of a deep learning-based \textbf{lung cancer classification system} not only 
showcases the power of artificial intelligence in modern healthcare but also aligns with several 
core objectives of the \textbf{United Nations Sustainable Development Goals (SDGs)}.

\begin{itemize}
    % SDG 3: Health and well-being
    \item \textbf{Goal 3 – Ensure healthy lives and promote well-being for all at all ages:} By 
    enabling earlier and more accurate detection of lung cancer through automated imaging analysis, 
    this project contributes directly to improving health outcomes and reducing mortality. Early 
    diagnosis can significantly enhance the effectiveness of treatment, improving patients' quality 
    of life and life expectancy.
    
    % SDG 9: Innovation and infrastructure
    \item \textbf{Goal 9 – Build resilient infrastructure, promote inclusive and sustainable 
    industrialization and foster innovation:} This AI-powered solution exemplifies how innovative 
    technologies can transform traditional healthcare systems. The use of robust, scalable deep 
    learning models within an accessible framework encourages the development of intelligent medical 
    infrastructure that can be adapted globally, especially in resource-constrained regions.
    
    % SDG 17: Collaboration and partnerships
    \item \textbf{Goal 17 – Strengthen the means of implementation and revitalize the global 
    partnership for sustainable development:} Collaboration between data scientists, healthcare 
    providers, and international research communities is vital. This project encourages 
    interdisciplinary cooperation and open-source contributions, promoting shared knowledge and 
    collective advancement toward better global healthcare systems.
\end{itemize}

\newpage

% Final wrap-up paragraph
By addressing these SDGs, the project not only advances AI in medicine but also actively supports 
the creation of a \textbf{healthier, more innovative, and cooperative global society}.

% Final concluding statement of the entire section
In conclusion, the proposed deep learning system provides a promising foundation for AI-assisted 
lung cancer screening. With further development and validation, such systems can serve as powerful 
diagnostic aids in clinical workflows, ultimately contributing to earlier detection and better 
patient outcomes.
